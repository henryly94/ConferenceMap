%%==================================================
%% abstract.tex for SJTU Bachelor Thesis
%% version: 0.5.2
%% Encoding: UTF-8
%%==================================================

\begin{abstract}

学术搜索是指针对目前学术论文数据量十分巨大,难以直接获取所需要的文献的情况下,在现有会议论文数据中建立合适的索引和联系,以结构化的信息方便查找选择。类似传统搜索引擎,为用户提供关键字或分类查找等方便的搜索方式。

而与传统搜索引擎使用者不同的是,学术搜索引擎使用者往往需要对整个领域有一个较为宏观的掌握才能够比较有效的使用搜索功能查找所需要的文献。而在同一会议中发表的论文之间,往往具有较强的相关性。但即使在一个特定领域,也存在复数个会议。如何在会议之间进行论文的推荐就是比较困难的问题。目前国内外的这一领域里,均缺少比较深入的研究和完善的解决方案。即使是在大数据十分火热的现今,对于学术大数据的特殊研究,特别是会议之间的研究获得的关注并不多。而对于很多研究者来说,对于某个会议的论文的研究,或者是同一领域内数个会议之间论文的比较,会带来对研究领域的 比较清晰和完备的认识,对研究带来收获。

在现有的学术论文数据基础上,主要考虑利用会议之间文章的相似度来定义会议之间的关系,基于各种自然语言处理方法,例如词嵌入以及神经网络语言模型等NLP方法来对文章进行分析,再辅助以例如引用以及作者之间关系等信息,来实现协同过滤的推荐方法。

  \keywords{学术搜索,数据挖掘,词嵌入, 自然语言处理, 推荐系统}
\end{abstract}

\begin{englishabstract}

Academic search aims to condition that current academic papers data set can be very large and will be difficult to obtain the necessary documents directly. Establishing the appropriate index and contacting to the structured information will be the solutions. Similar to the traditional search engine, to provide users with keywords or classification search and other convenient search methods.

Unlike traditional search engine users, academic search engine users often need to have a more general grasp of the entire field in order to be more effective when search the required paper. Paper published in the same meeting often have a strong correlation. But even in a particular area, there are plural meetings. So the problems here will be how to recommend a paper between similar meetings. In this area, there are few in-depth researches and solutions. 

On the basis of the existing academic papers, we mainly consider using the similarity between articles to define the relationship between meetings, based on various natural language processing methods, such as word embedding, neural network language model and other NLP approach. In the auxiliary, we try to use some collaborative filtering recommended method to improve the performance.

  \englishkeywords{\large acadamic search, BigData, word embedding, NLP, recommendation system}
\end{englishabstract}
