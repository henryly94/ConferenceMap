%%==================================================
%% conclusion.tex for SJTU Bachelor Thesis
%% version: 0.5.2
%% Encoding: UTF-8
%%==================================================

\chapter*{全文总结\markboth{全文总结}{}}
\addcontentsline{toc}{chapter}{全文总结}

\section{研究成果}

本文主要介绍从头实现一个学术大数据推荐系统的过程.该推荐系统利用了学术大数据中的会议关系并且将其可视化了,以此来提供了更加个性化的推荐服务.

可视化及推荐系统仅仅分别是本课题中采用的技术的一个应用的方面.本文的主要贡献是实现了学术会议基于向量化方法的特征提取.将自然语言处理中的N-gram语言模型与学术论文数据结合起来,获得了一种全新的学术论文以及学术会议的向量表示.

与被广泛采用的LDA等文本聚类模型不同的是,由于采用了分维度的聚类作为前置操作,本文中所采用的向量化方法最终得到的无论是文章向量还是学术会议向量,其每个维度都具有很好的可读性和可解释性.而同时,向量整体之间的关系也十分符合现实中的学术会议,论文之间的关系,可以很轻松的在目前得到的向量基础上进行二次操作,例如再聚类等等.

\section{下一步工作}

本课题未来的研究重点主要集中于如何在保持现有向量化结果的前提下,能够构建出一个评估方法对结果进行评价和修正.

可以发现,虽然本文中的方法虽然具有很好的解释性,但得到的向量表示却缺少一个可以评估其有效性以及正确性的评价体系.这往往也是各个自然语言处理方法中所欠缺的部分.常见的方法有求在原数据集上的混乱度以及交叉熵等等.

由于本文中实现的向量为分布式的聚类表示,并且由余弦相似度来度量其距离,所以考虑可能采用流型学习在其对应高维超球面上取近似可能是个合理的手段.考虑到对应的文章为会议上发表的学术论文这一性质,还可以利用其中的作者关系以及文章之间的引用关系来进行辅助.

另外,未来的工作还会更深入地研究这种特征提取技术在学术论文中的更多应用,设计出各种可能的使用场景,来更好的发掘出这种技术的实用价值.