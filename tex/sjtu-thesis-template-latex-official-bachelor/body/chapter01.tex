%%==========================
%% chapter01.tex for SJTU Bachelor Thesis
%% version: 0.5.2
%% Encoding: UTF-8
%%==================================================

%\bibliographystyle{sjtu2} %[此处用于每章都生产参考文献]
\chapter{绪论}
\label{chap:introduction}

\section{学术搜索现状简介}
\label{sec:academic-intro}

网络信息时代下,各个学科的学术研究成果的交流自然而然变得更加频繁且不可或缺.其中在我们日常研究生活中最为经常应用到的部分就是灌注了其他研究者智慧果实和研究成果的学术论文.获取一篇想要阅读的论文通常需要自费购买或者依靠于对应的学术机构集体购买.例如IEEE的数据库\footnote{\url{http://ieeexplore.ieee.org}}以及国内的中国知网\footnote{\url{http://cnki.net}}.当然也有类似arXiv\footnote{\url{https://arxiv.org}}的网站可以免费阅读下载上面的论文. 然而这些数据库的问题在于其中存储的论文数据相互之间几乎没有交集.导致如果要寻找某一篇论文有时候必须遍历数个数据库才能找到具体想要的论文.这样的行为明显增加了学术研究中不必要的时间开销.而类似的情况以前也曾经出现过,那就是在互联网的发展的早期,各个网站之间并不完全相通,缺乏一个合适的流量引导的工具导致网络的相对闭塞.而门户网站以至于之后的搜索引擎的出现正是填补了这个空缺,让整个网络真正的互相连接起来了,也方便了在互联网上浏览的所有用户.而在网络中的学术论文这个领域,相对应出现的工具就是学术搜索引擎了.

在当今世界上,被应用得比较广泛,使用者数量比较多的学术搜索引擎应该要数美国Google公司开发的Google Scholar网站\footnote{\url{https://scholar.google.com}}.这个网站也是目前公开的,数据量最大,最方便的学术搜索引擎.这个搜索引擎的使用方式和普通的网站搜索引擎类似.支持按照关键词搜索,通过作者检索,通过高级搜索进行更加精确地搜索等功能\cite{夏旭2006基于}. Google Scholar也提供了学术论文作者的个人资料页面,其中可以看到该作者相关的很多信息,包括按年份划分的被引用数以及一些统计数据等.例如 \textbf{h-index},意义是被引用数超过该数字次数的文章数.从这些个人页面也可以很快地获得对此人研究成果的大致认知.可以说也从侧面加强了网站整体的功能性以及易用性.

与Google Scholar相似的,还有微软公司的微软学术搜索\footnote{\url{http://cn.bing.com/academic}}.是由微软亚洲研究院(MSRA)开发的免费学术搜索引擎。特点是使用了对象级别的信息检索途经,可以通过微软自己的数据库来找到某个学术领域内的领先的研究者和顶尖的会议和期刊,甚至还可以研究领域兴盛和发展的具体过程.还可以用来发现一些学术研究学科内的研究热点和正在上升期的学术新星等.虽然这个搜索引擎目前主要收录了与计算机科学和信息科学相关的学术数据,但这些更为高级的应用功能无疑是对于用户的使用体验有很大提高的.

除了这些国外公司开发的学术搜索引擎以外,国内也有高校开发出自己的学术搜索引擎,例如清华大学的AMiner\footnote{\url{https://aminer.org}}和上海交通大学的Acemap\footnote{\url{http://acemap.sjtu.edu.cn}}. 这些高校搜索引擎的特点是虽然数据量无法和Google Scholar那样的大体量比肩,但开发者更注重用户在使用学术搜索引擎时可能需要的一些个性化的功能.例如Acemap之中,用户可以通过可视化的方式,由大及小很轻松的选择出自己感兴趣的学术研究领域,来检索其中的论文.更可以通过合作者地图,学术机构地图等方式来贴合各种个性化的需求,来提高寻找所需要的学术论文的效率.


\section{系统简介}
\label{sec:nlp-intro}

由于在实际的学术研究中,相对于关注某个知名研究者或者研究机构的研究成果,我们往往会更加关注某一个领域的过往以及最新的研究成果也即是在该领域发表的学术论文.而且不同的人关注的领域往往并不会宽泛到某个一级学科甚至于二级学科,而是具体到某个很细分的方向.例如对于同样是关注机器学习这个领域的研究者,有些人关注的重点在于监督学习而有些人则会对非监督学习更为感兴趣.还有现在大火的增强学习,也是一个截然不同的领域.更进一步地,在相似的一些算法上,也有不同的侧重点在.这样就给我们在寻找一系列感兴趣的论文时增加了难度.仅仅依靠几个关键词无法很好的找到所需要的论文,更无法发现一些提出的崭新的思想和概念.而通过作者和合作者甚至于引用关系来检索也有很大的局限性,因为每个研究者的研究兴趣和方向也是会随着时间的变化而有相应的变化,无法仅仅通过锁定数个研究者来获得感兴趣的方向和领域的论文.很自然地,我们就产生了一个想法,如何通过现有的数据,来改善我们的学术搜索使用的体验,来更好的去切合我们去查找某个细分领域内所有优质论文的需求呢?

这乍一看是一个并没有表面上那么简单的想法.论文的思想和方向无法简单地通过数个关键词来定义.而具体涉及到论文的内容,要通过自然语言处理的方法来直接进行文本分类,来进一步地将我们所寻找的相似领域的论文来划分到一起,先不提其效果比较难以检验,学术论文作为研究者的思维结晶和学术成果,作为文本,它的单位信息量是大于我们常见的新闻,小说等文本类型的.而且论文文本之中结构也十分复杂,为了能够很好地将自己的研究成果向其他研究者阐述清楚,研究者们往往需要在学术论文之中附上大量的图标,数据,算法.这些并不标准的部分也增加了分析学术论文整个文本的难度.目前为止,有关于学术论文文本分类的研究比较少,想必也有这个方面的原因.所幸这个想法能够很自然地落到实处,通过自然语言处理方法来进行对应的学术论文分类并不是一个非常糟糕的选择.但具体应用到学术论文上时,我们考虑采取一些其他的手段来辅助我们的分类.

首先考虑到学术论文的特殊性在于其结构化较为明显,大部分发表的学术论文,基本上都包含了摘要,绪论,正文等几个部分.这也体现了学术论文与其他文本的区别的一点在于其作者都是经过了充分训练的科学工作者,其整体结构的严谨性自然能够得到保证.其次则是学术论文大多通过会议和期刊来发表.我们很自然地想到,在学术论文从写作到投稿这一过程中,研究者对于投稿的期刊和会议的选择,其实就是一次人为的标签行为.因为不同的期刊和会议,所接受的论文的研究主题和范围都是有所限定的.虽然有时候会随着时间的经过,世界上研究热点的变化而稍有变化,但依然能作为一个判定地依据.特别是学术会议,作为一个有时效性的活动,其中发表的学术论文的选题和方向通常都具有很强的时代性和针对性.与此相对的是许多期刊因为是长期发表的刊物,其内容受时间和时代的影响就比较明显.任何一个期刊里刊登在2016年的文章和2006年的文章关注的热点肯定是截然不同的.很显然,与其直接研究学术论文文本的分类,研究学术会议整体之间的关系,再通过相似的会议来进行学术论文的推荐就是一个比较容易且高效的方法.

那么对于一个了解数据挖掘的人来说,这个问题就集中到如何去在机器能够认识的领域内去定义一个学术会议了,简单地来说就是如何对学术会议去做特征提取.这时候又有一个很自然的想法,既然学术会议是其中学术论文的一个隐含的特征,如果我们把学术会议看做是学术论文的集合,那么我们显然可以通过其中包含的学术论文反过来定义会议本身.如何定义一篇学术论文呢?常见的文本特征抽取的方法有从简单的词频,\textbf{TF-IDF}到\textbf{N-gram}算法,模拟退火算法等较为复杂的方法.这里我们考虑到我们最终的目的在于理清学术会议之间一个相对的关系,注重的是相对.所以我采取了具有很好线性特征的Word2vec\cite{mikolov2013distributed,mikolov2013efficient}来实现最终的文本向量化.

具体的做法是利用单层的神经网络语言模型\textbf{Skip-gram}来训练词向量$\mathbf{V}_{word}$,再将训练好的词向量进行聚类.利用聚类来对文本向量化.聚类算法也有很多中选择,包括层次化聚类算法,划分式聚类算法以及基于网格和密度的聚类算法等\cite{孙吉贵2008聚类算法研究},这里我们选择使用划分式聚类算法中的\textbf{K-Means}算法来进行聚类.经过聚类之后的词向量,每个类型都代表了学术论文中比较接近的一类词语,这时候我们只需要简单的统计出不同类型的词语的词频,就可以获得代表文章的向量$\mathbf{V}_{doc}$.

得到了文本向量之后,我们将进一步考虑利用文本向量去生成对应会议向量.这时候我们主要考虑到会议向量应该代表了该学术会议中学术论文的主要趋势,以及会议向量之间能够通过某种方式来比较之间的相关程度.在上一步中,我们采用了不同类型词语的词频来代表文本向量,那么将一个学术会议中所有学术论文的文本向量取平均,然后通过余弦相似度来进行比较就是一个很容易想到的方法.实际上这种方法也确实能够得到非常令人满意的结果.





